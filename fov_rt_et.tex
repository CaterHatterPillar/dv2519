% fov_rt_et.tex
% ---------------------------------------------------------------------------
% Author guideline and sample document for EG publication using LaTeX2e input
% D.Fellner, v1.13, Nov 13, 2007

\documentclass{egpubl}

% --- for  Annual CONFERENCE
\ConferenceSubmission % uncomment for Conference submission
% \ConferencePaper      % uncomment for (final) Conference Paper
% \STAR                 % uncomment for STAR contribution
% \Tutorial             % uncomment for Tutorial contribution
% \ShortPresentation    % uncomment for (final) Short Conference Presentation
%
% --- for  CGF Journal
% \JournalSubmission    % uncomment for submission to Computer Graphics Forum
% \JournalPaper         % uncomment for final version of Journal Paper
%
% --- for  CGF Journal: special issue
% \SpecialIssueSubmission    % uncomment for submission to Computer Graphics Forum, special issue
% \SpecialIssuePaper         % uncomment for final version of Journal Paper, special issue
%
% --- for  EG Workshop Proceedings
% \WsSubmission    % uncomment for submission to EG Workshop
% \WsPaper         % uncomment for final version of EG Workshop contribution
%
 \electronicVersion % can be used both for the printed and electronic version

% !! *please* don't change anything above
% !! unless you REALLY know what you are doing
% ------------------------------------------------------------------------

% for including postscript figures
% mind: package option 'draft' will replace PS figure by a filname within a frame
\ifpdf \usepackage[pdftex]{graphicx} \pdfcompresslevel=9
\else \usepackage[dvips]{graphicx} \fi

\PrintedOrElectronic

% prepare for electronic version of your document
\usepackage{t1enc,dfadobe}

\usepackage{egweblnk}
%\usepackage{cite}

% For backwards compatibility to old LaTeX type font selection.
% Uncomment if your document adheres to LaTeX2e recommendations.
\let\rm=\rmfamily    \let\sf=\sffamily    \let\tt=\ttfamily
\let\it=\itshape     \let\sl=\slshape     \let\sc=\scshape
\let\bf=\bfseries

% end of prologue


\usepackage{gensymb} % \degree\ command.
\usepackage{algorithm} % Algorithm environment for pseudocode.
\usepackage{algpseudocode} % Pseudocode package.
\usepackage{graphicx} % PNG figures.
\usepackage{hyperref} % Urls.
\usepackage{float} % Floaty fine figures.
\usepackage{stfloats} % To allow placement of float* environment at bottom of page.
\usepackage{pgf} % Used to round numbers to given number of significant figures.

\newcommand*{\no}[1]{%
    \pgfmathprintnumber[
        fixed,
        precision=3,
        fixed zerofill=true,
        ]{#1}}%

\restylefloat*{figure}
\restylefloat*{table}

\let\OLDthebibliography\thebibliography
\renewcommand\thebibliography[1]{
  \OLDthebibliography{#1}
  \setlength{\parskip}{0pt}
  \setlength{\itemsep}{0pt plus 0.3ex}
}

\title[Foveated Real-time Ray Tracing]{Foveated Real-time Ray Tracing}

\author[E. Nilsson]
       {E. Nilsson$^{1}$\\
         $^1$Blekinge Institute of Technology, Karlskrona, Sweden
       }

% if the Editors-in-Chief have given you the data, you may uncomment
% the following five lines and insert it here
%
% \volume{27}   % the volume in which the issue will be published;
% \issue{1}     % the issue number of the publication
% \pStartPage{1}      % set starting page

\begin{document}

\maketitle

% fov_rt_et_ABSTRACT.tex
\begin{abstract}
During development of a DirectX ray tracer, computational complexity is deemed too high for target system specifications.
Accordingly, foveation - having been used to improve rasterization performance in the past - is deemed appropriate for the purposes of optimizing performance without impairing perceived scene quality.
Using a Tobii eye tracker, we successfully optimize rendering to reduce frame time by over $90\%$.

We present an open-source implementation of foveated ray tracing, attesting that the performance benefits of foveation demonstrated by peers in rasterization extends to ray tracing.
This paper presents an overview of the devised algorithm, which is open-source, along with results garnered during an experiment comparing the foveated implementation to the non-foveated algorithm.
Furthermore, the authors' conclusions and thoughts on future work in the area is presented along with information on how to retrieve the solution for further study.

\begin{classification} % according to http://www.acm.org/class/1998/
\CCScat{Computer Applications}{J.7}{Computers in Other Systems}{Real Time}
\CCScat{Computer Graphics}{I.3.7}{Three-Dimensional Graphics and Realism}{Raytracing}
\end{classification}

\end{abstract}

% fov_rt_et_INTRODUCTION.tex
\section{Introduction}
Ray tracing has long since been used for computer graphics, both in offline and real-time scenarios, some real-time use-cases impressively stretching as far back as a 1987 CAD-application~\cite{stay87}.
Recently, the fast progression of modern processors have accomodated for more extensive use of ray tracing technologies in real-time, as demonstrated by Intel Corporation with their reseach project Quake Wars: Ray Traced in 2009~\cite{pohl09}, an implementation of the Id Software video game Enemy Territory: Quake Wars, popularizing the idea of ray tracing renderers for consumer video games.

In 2012, Garc\'ia et al. presented a ray tracing model using DirectCompute~\cite{garcia12}; thus utilizing modern throughput-oriented many-core devices for accomplishing ray tracing in real-time.
During implementation of a similar algorithm for a university project, we discovered that our video card was not keeping up to speed.
Hence, we required a faster algorithm in order to ease development.
We thus decided to accelerate the rendering using foveation to reduce the GPU workload.

% fov_rt_et_BACKGROUND.tex
\section{Background}
\subsection{Ray tracing}
Ray tracing is an image synthesis technique in which rays are used to determine scene- and geometry visibility.
In computer graphics, rays often originate from the observer view-point based on the framebuffer texel grid in order to determine framebuffer colors.
Each ray subsequently identifies the closest scene element it may intersect, from which lighting, shadows, and other effects may be computed.

On the contrary, modern real-time graphics commonly use the rendering technique of rasterization.
Rasterization technologies, which are often significantly faster than ray tracing methodologies to map scene geometry to a computer screen, often require complicated methods to approximate (often roughly) effects such as shadows or reflections; effects that might be described as trivial in terms of ray tracing.
Yet, rasterization techniques are often preferred to ray tracing renderers due to their superior speed and 'good-enough' results.

Some of the performance gains rasterization methodologies often have over ray tracing is that of efficient data coherence behaviors, inducing high computational complexity.
While the rasterization process may cache and share computed results, ray tracing techniques commonly eveluate each ray individually; especially considering each ray may collide with different geometry.
Furthermore, complex geometry stresses the importance of acceleration structures for real-time ray tracing.
While acceleration structures, such as BVH trees, may offer great performance improvements, high-resolution render targets, such as emergent 4k resolutions, may come at high initial costs.

\subsection{Eye tracking}
Eye tracking is the process of sampling an observers gaze, often in relation to a surface such as a computer screen.
For the purposes of eye tracking, one may use an eye tracker, which is a device used to establish and measure positions of an observers eyes\cite{duchowski07}; such as \textit{fixations} and intermediate \textit{saccades}\cite{rayner98}.
The most common eye tracking device type used today is possibly the optical eye tracker; which uses a sequence of video images to measure movement of the eyes\cite{duchowski07}.
Recently, eye tracking technologies - although not a new subject to researchers - have become more accessible to the general public, mayhaps most notably with the recent Tobii EyeX Controller devkit.

\subsection{Limitations in human vision}
Human vision may be thought of as being subdivided in three distinct zones; the foveal, parafoveal, and peripheral vision\cite{rayner98}.
The fovea, which makes out roughly $2$\degree\ of human eye-sight, is the area in which the human eye features the outmost acuity due to vast concentrations of visual receptors.
As such, we position the fovea on the stimulus we wish to observe.
Along the outer reaches of the fovea, and extending $5$\degree\ from the center of the fovea, lie the parafovea which, while still quite acute, is considerably less keen than the fovea.
The remaining (majority) of the human vision we may classify as peripheral vision which, while not acute as fovea or parafovea, constitutes the remaining visual field of roughtly $135$\degree\ vertical- and $160$\degree\ horizontal vision\cite{guenter12}.
We may refer to the decrease of the receptive qualities of human eye-sight to foveation.

\subsection{Foveated rendering}
In computer graphics aforementioned eye-sight limitations are often ignored; real-time graphics applications such as video games rendering complex scenes at full-resolution independently of where the observer is looking.
One might argue that vast amounts of processing power is spent on rendering high resolution imagery for human peripheral vision which may not necessarily appreciate high image quality.
After all, a $5$\degree\ foveal area is likely a small component of most display systems.

Hence, in real-time graphics where time is of the essence, it may be benificial to only render those parts of a framebuffer which the eye may appreciate, at high quality; reducing rendering complexity for a potentially large area of the display.
Utilizing foveation in image processing is not a new idea\cite{levoy90} yet one that seems to have escaped developers of consumer real-time graphics, possibly due to the lack of low-latency eye tracking devices in consumer markets.

% fov_rt_et_CONTRIBUTION.tex

\section{Contribution}

\begin{figure*}
\parbox{.5\linewidth}{%
\centering%
\input{gnu_non-foveated.tex}
\caption{Non-foveated frametimes (ms).}
\label{fig:histogram_non-foveated}}
\hfill%
\parbox{.5\linewidth}{%
\centering%
\input{gnu_foveated.tex}
\caption{Foveated frametimes (ms).}
\label{fig:histogram_foveated}}
\end{figure*}

\begin{table*}[bp]
\parbox{.5\linewidth}{
\centering
\begin{tabular}{l|rrrr}
Stage & Avg. & Std. & Min. & Max.     \\ \hline
Gen. Rays & \no{1.3688513} & \no{0.02400363344} & \no{1.26811} & \no{1.50389} \\
Intersect & \no{12.3195429} & \no{0.001177461679} & \no{12.3164} & \no{12.3252} \\
Color & \no{18.0143106} & \no{0.4686627975} & \no{16.8978} & \no{22.0308}
\end{tabular}
\caption{Non-foveated stages elapsed time (ms).}
\label{tab:nonfoveated}}
\hfill
\parbox{.5\linewidth}{
\centering
\begin{tabular}{l|rrrr}
Stage & Avg. & Std. & Min. & Max. \\ \hline
Gen. Rays & \no{0.08972366} & \no{0.001852071991} & \no{0.0851111} & \no{0.120556} \\
Intersect & \no{1.05272049} & \no{0.02949215224} & \no{1.04833} & \no{1.87911} \\
Color & \no{1.72431991} & \no{0.07007683999} & \no{1.56156} & \no{3.15844}
\end{tabular}
\caption{Foveated stages elapsed time (ms).}
\label{tab:foveated}}
\end{table*}

The ray tracing renderer devised for the purpose of this study adheres to the model presented by Garc{\'i}a et al.~\cite{garcia12}: devised in DirectCompute using C$++$.
The algorithm renders the scene using three distinct steps that compiles a number of rays equal to that of the application window resolution, performs intersections tests on each ray with scene geometry to establish the closest scene element, and finally draws the scene to the framebuffer using scene lights- and geometry in order to establish what areas of the scene are in shadow.
The ray tracing algorithm is presented as pseudocode in algorithm \ref{algrt}.
Note that some details are not presented in the algorithm in order to keep the pseudo code concise.

\begin{algorithm}
\begin{algorithmic}[1]
\Procedure{raytrace}{$rays, reflCnt$}
\caption{Ray tracing algorithm}\label{algrt}
\State $rays\gets$\Call{GenRays}{screen, frustum}
\While{$reflCnt>0$}
\ForAll{$rays$}\Comment{for each pixel}
\If {\Call{intersects}{ray, objs}}
    \State $obj\gets objs$
\EndIf
%\Require $obj$ % If you're to be entirely precise.
\ForAll{$lights$}
\State $color\gets$\Call{Shadow}{ray, obj}
\State $color\gets$\Call{Lighting}{obj}
\EndFor
\State $backbuffer\gets color$
\EndFor
\State $reflCnt\gets reflCnt - 1$
\EndWhile
\EndProcedure
\end{algorithmic}
\end{algorithm}

In order to optimize this algorithm to make it run sufficiently fast on the platform at hand we made use of a Tobii EyeX Dev Kit to establish where on the screen the observer is focusing his or her gaze.
Using this information, we may render only parts of the application window - where the user has his or her gaze fixation - at full resolution; rendering those areas of the window in the user's peripheral vision at a lower resolution.
Since, as described in algorithm \ref{algrt}, the total number of computed rays is based off the texel grid of the framebuffer this may drastically reduce the number of rays required to render a scene.

\subsection{Hardware}
For the purpose of this experiment, we make use of the Tobii EyeX Devkit Controller, which is a consumer-level corneal-reflection eye tracking device, which may the position the gaze of an oberserver on a computer screen.
The EyeX controller, while still a developer's prototype variant, is a fairly competent device; the consumer version expected to be released later this year.

The experiment is performed on a Windows~8.1 system with the following specifications:
\begin{itemize}
\setlength\itemsep{0em}
\item Intel Q9550 Quad Core 2.83GHz
\item ATI Radeon HD 5800
\end{itemize}

For the screen, we use a $23''$ $510$mm~$\times$~$287$mm Samsung~Syncmaster~$2343$ monitor, beneath which the eye tracking device is placed for the duration of the experiment.

\subsection{Software}
For retrieval of gaze positional data and communication with the eye tracking device, we utilize the Tobii C/C$++$~SDK.

In order to achieve varying levels of quality - or resolution - in the field-of-vision, the rendering process is subdivided into a number of 'FOV's (short for fovea, parafovea, etc.); each it's own render target.
Using an arbitrary of number these FOVs, we may vary the quality of the rendered scene accross the user's field of vision.
For the purpose of this study, although FOVs for both foveal-, parafoveal-, and peripheral vision have been prototyped, we limit the variance in quality to that of parafoveal and peripheral vision.
Thus, we render the scene at two different resolutions during the experiment; corresponding to parafoveal- and peripheral vision.

This is due to the EyeX controller device.
Being a consumer-level developer's kit, the device tracking may bring about a jittery render target when the designated area is as small as $2$\degree\ of the observer's view.
As this is considerably less noticable with a larger area (giving the eye tracking device more time to garner gaze positional data when the user moves his or her eyes); rendering the entirety of the parafoveal vision (5\degree ) at high quality is deemed more appropriate for the equipment at hand.
After all, the purposes of foveated rendering is to reduce computational complexity whilst maintaining percieved scene quality.

For our setup, the observer is positioned roughly 700mm from the screen (which is the distance relayed appropriate by the device).
The application window resolution is that of $1152\times 1152$, a resolution chosen due to it being the highest our system setup could display whilst keeping the resolution square.
A $5$\degree\ parafovea thus bring about an area of roughly $31$mm$\times $$31$mm, which represents, for the utilized screen, roughly $123\times 123$~pixels; a small percentage of the application window resolution.
Note, however, that this component would be considerably smaller should the application run at even higher resolutions.

For the peripheral vision, or the low-quality render target, we chose to render the scene at only a fourth of the full-resolution.
Accordingly, our rendering algorithm is performed in an additional two steps; wherein the scene is rendered at full-resolution in a grid of only $123\times 123$ pixels, while the rest of the framebuffer is rendered at a fourth of full-resolution and subsequently upscaled to fit the application window.
The high-resolution parafoveal framebuffer is then copied onto the screen framebuffer at the location where the observer is currently focusing his or her gaze.

See figure~\ref{fig:fov} for a sample image of the devised foveated scene.
Note that the surrounding scenery, including the cube's reflection in the plane, is rendered at a lower resolution.
This particular scene, on the contrary to the experiment, renders the peripheral areas in an eighth of the full-resolution in order to make the foveated effect more prominent and noticable for the sake of visualization.

% fov_rt_et_CONCLUSION.tex

\section{Conclusion}
\ldots

% Desribe the scene
% Vertex count
% three reflections

% Describe experiment
% Fixation of foveated area
% scene
% Performed 1000 times
% Mean values presented in table

% Over 90% reduction in all stages

\begin{table}[h]
\begin{tabular}{lll}
  & Non-foveated & Foveated \\
  Generate Rays & 1.36\phantom{0} ms & 0.09 ms \\
  Intersection & 12.32 ms & 1.05 ms \\
  Lighting & 18.01 ms & 1.72 ms
\end{tabular}
\caption{Mean elapsed time in milliseconds for the ray tracing shader stages for non-foveated and foveated samples. Note that the latter two shader stages are performed multiple times (one per reflection); the elapsed times of which are concatenated for these samples.}
\end{table}

\subsection{Future work}
\ldots

% Problems that real-time foveated ray tracing faces:
% Peripheral anti-aliasing issues
% Eye tracking latency

%% Avg:
%% 0.0897236674
%% 1.05272049
%% 1.72431991
%% EVE:res caterpillar$ python avg.py 

%% Avg:
%% 1.3688513
%% 12.3195429
%% 18.0143106


\bibliography{fov_rt_et}{}
\bibliographystyle{eg-alpha}

\end{document}
