% fov_rt_et_INTRODUCTION.tex
\section{Introduction}
Ray tracing has long since been used for computer graphics, both in offline and real-time scenarios, some real-time use-cases impressively stretching as far back as a 1987 CAD-application\cite{stay87}.
Recently, the fast progression of modern processors have accomodated for more extensive use of ray tracing technologies in real-time, as demonstrated by Intel Corporation with their reseach project Quake Wars: Ray Traced in 2009\cite{pohl09}, an implementation of the Id Software video game Enemy Territory: Quake Wars, popularizing the idea of ray tracing renderers for consumer video games.

In 2012, Garc\'ia et al. presented a ray tracing model using DirectCompute\cite{garcia12}; thus utilizing modern throughput-oriented many-core devices for accomplishing ray tracing in real-time.
During implementation of a similar algorithm for a university project, we discovered that our video card was not keeping up to speed.
Hence, we required a faster algorithm in order to ease development.
We thus decided to accelerate the rendering using foveation to reduce the GPU workload.
