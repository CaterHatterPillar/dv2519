% fov_rt_et_ABSTRACT.tex
\begin{abstract}
During development of a DirectX ray tracer, computational complexity is deemed too high for target system specifications.
Accordingly, foveation - having been used to improve rasterization performance - is deemed appropriate for the purposes of improving performance without impairing percieved scene quality.
Using a Tobii eye tracker, we successfully optimize rendering to reduce frametime by over $90\%$.

We present an open-source implementation of foveated ray tracing, attesting that the performance benefits of foveation demonstrated by peers in rasterization extends to ray tracing.
This paper presents an overview of the devised algorithm, which is open-source, along with results garnered during an experiment compairing the foveated implementation to the non-foveated algorithm.
Furthermore, the authors' conclusions and thoughts on future work in the area is presented along with information on how to retrieve the solution for further study.

\begin{classification} % according to http://www.acm.org/class/1998/
\CCScat{Computer Applications}{J.7}{Computers in Other Systems}{Real Time}
\CCScat{Computer Graphics}{I.3.7}{Three-Dimensional Graphics and Realism}{Raytracing}
\end{classification}

\end{abstract}
