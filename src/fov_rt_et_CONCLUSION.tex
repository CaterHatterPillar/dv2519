% fov_rt_et_CONCLUSION.tex

\section{Conclusion}
The scene used for the purpose of this experiment is a mesh consisting of $12$ vertices that are indexed to create $14$ triangles, which are scaled, rotated, and translated into the scene.
This mesh, although positioned in a different manner in relation to the camera, is presented in figure~\ref{fig:fov}.
Furthermore, the scene is lit using three pointlights; everyone of which may cast shadows.

In order to render the scene, which rotates every given frame, the renderer performs three reflections thus repeating the intersection- and lighting stages of the algorithm three times per pixel.
During the experiment, the higher-resolution parafoveal render target was locked to be positioned in the center the screen.
As such, the eye tracking had no influence on the results.

We collected elapsed times for each stage, concatenated if a stage was performed multiple times (for intersection- and lighting stages), during $1000$ frames.
The mean values of these measurements are presented in milliseconds in table~\ref{tab:res}.

\begin{table}[h]
\begin{tabular}{lll}
  & Non-foveated & Foveated \\
  Generate Rays & 1.36\phantom{0} ms & 0.09 ms \\
  Intersection & 12.32 ms & 1.05 ms \\
  Lighting & 18.01 ms & 1.72 ms
\end{tabular}
\caption{Mean elapsed time in milliseconds for the ray tracing shader stages for non-foveated and foveated samples. Note that the latter two shader stages are performed multiple times (one per reflection); the elapsed times of which are concatenated for these samples.}
\label{tab:res}
\end{table}

From the measurements presented in table~\ref{tab:res}, we may establish that foveation reduced execution time of over $90\%$ in all three ray tracing stages.
This reduction was significant enough to run the ray tracer in high-definition resolutions, and thus made it possible for us increase the application resolution to above that of $800\times 800$ which we were previously limited to.

Furthermore, the foveated rendering is mostly transparent and not too noticable.
Although gaze position latency and aliasing issues in the low resolution render target (in the peripheral vision) is noticable, it does not dispruct the observer's perception of the scene.
That being said, foveation accomodating for considerably better performance is a good-enough compromise.

\subsection{Future work}
\ldots

% Problems that real-time foveated ray tracing faces:
% Peripheral anti-aliasing issues
% Eye tracking latency and accuracy
